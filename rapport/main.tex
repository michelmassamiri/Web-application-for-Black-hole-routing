\documentclass[11pt]{report}
\usepackage[a4paper]{geometry}
\usepackage[myheadings]{fullpage}
\usepackage{fancyhdr}
\usepackage{lastpage}
\usepackage{graphicx, wrapfig, subcaption, setspace, booktabs}
\usepackage{rotating}
\usepackage{diagbox}
\usepackage[T1]{fontenc}
\usepackage[font=small, labelfont=bf]{caption}
\usepackage[protrusion=true, expansion=true]{microtype}
\usepackage{sectsty}
\usepackage{url, lipsum}
\usepackage{mathptmx}
\usepackage[utf8]{inputenc}
\usepackage[francais]{babel}
\usepackage{diagbox}
\usepackage{hyperref}
\bibliographystyle{alpha}
\pagestyle{plain}


\newcommand{\HRule}[1]{\rule{\linewidth}{#1}}
\setcounter{tocdepth}{5}
\onehalfspacing
\setcounter{secnumdepth}{5}
\renewcommand{\thesection}{\arabic{section}}

\usepackage{xargs}                      % Use more than one optional parameter in a new commands
\usepackage[pdftex,dvipsnames]{xcolor}  % Coloured text etc.
% 
\usepackage[colorinlistoftodos,prependcaption,textsize=tiny]{todonotes}
\newcommandx{\unsure}[2][1=]{\todo[linecolor=red,backgroundcolor=red!25,bordercolor=red,#1]{#2}}
\newcommandx{\change}[2][1=]{\todo[linecolor=blue,backgroundcolor=blue!25,bordercolor=blue,#1]{#2}}
\newcommandx{\info}[2][1=]{\todo[linecolor=OliveGreen,backgroundcolor=OliveGreen!25,bordercolor=OliveGreen,#1]{#2}}
\newcommandx{\improvement}[2][1=]{\todo[linecolor=Plum,backgroundcolor=Plum!25,bordercolor=Plum,#1]{#2}}
\newcommandx{\thiswillnotshow}[2][1=]{\todo[disable,#1]{#2}}
%

%-------------------------------------------------------------------------------
% TITLE PAGE
%-------------------------------------------------------------------------------

\begin{document}

\title
{
	\Large{}
	\HRule{2pt} \\ [0.5cm]
	\LARGE \textbf{\uppercase{PDP: Routage vers trou noir piloté à distance}}
	\HRule{2pt} \\ [0.5cm]
    	\normalsize \today
}

\date{}


\author
{
	\LARGE{Université de Bordeaux} \\
	\\
    BRISSET Rémi \\
    CHAUVEAU Pierre\\
    MASSAMIRI Michel \\
    PERUZZETTO Enzo \\
    \\
    \\
    \\
    \url{https://services.emi.u-bordeaux.fr/projet/git/blackholepdp}
}

\maketitle
\tableofcontents
\section{Présentation du projet}
\subsection{Intitulé}
 L'objectif de ce projet est de développer un outil permettant à un administrateur réseau de définir à distance à partir d'un client Web, des routes menant vers des trous noirs pour dévier des attaques réseaux. Ces routes seront envoyées à un serveur de route qui les diffusera auprès de tous les serveurs BGP du domaine. Le logiciel devra être implémenté en Javascript et du côté serveur il devra piloter le logiciel ExaBGP écrit en Python. L'application Web devra être de type RESTful et elle s'appuiera éventuellement sur un framework JS. Elle devra supporter le routage vers trou noir par la destination, par la source et par la communauté BGP.


\subsection{Le routage vers trou noir \cite{Cisco}} 

La déviation des routes vers un trou noir, aussi appelée "Remotely-Triggered Black Hole (RTBH)" en anglais, est une technique qui permet de faire tomber (suspendre) un trafic provenant d'une source étant indésirable, avant que ce dernier puisse entrer dans un réseau protégé.
\newline
Cette technique est appliquée sur un routeur BGP( Border Gateway Protocol )qui lui, utilise le protocole TCP afin d'échanger des informations de routage avec des autres routeurs BGP.
\\
\\
Le routage vers trou noir est essentiellement utilisé pour défendre ou proprement dit pour atténuer les attaques DDoS (distributed-denial-of-service). Les trous noirs sont placés principalement dans un réseau pour lequel, on peut dévier et/ou suspendre le trafic lorsque le système détecte une attaque. 
\newline
Pour que le système puisse dévier des router, il se base sur l'adresse IP de la destination ou bien de l'adresse IP source. Donc, il existe deux méthodologies :
\\
\begin{itemize}
\item \textbf{Destination-Based Remotely Triggered Black Hole Filtering } : On rend l'adresse IP de la destination inaccessible, en déviant toutes les routes allant à cet adresse vers le trou noir.
\item \textbf{Source-Based Remotely Triggered Black Hole Filtering }
: Dans ce scénario, si le trafic provenant d'une adresse IP est susceptible d'être une attaque, alors, tout trafic lié à cet adresse IP serait suspendu. Cela veut dire que selon l'adresse source IP, cette dernière ne peut pas avoir accès à sa destination. En outre, on fait tomber tous les chemins partants d'une adresse IP source précise.   
\end{itemize}



\section{Analyse de l'existant}
 \subsection{Erco.xyz \cite{Erco}}


Erco est un outil initialement développé à l'Université de Lorraine facilitant la configuration  des routes réseau avec Exabgp en réécrivant une partie du fichier de configuration d'Exabgp. Erco fournit une API RESTful et une interface web utilisateur. L'interface web permet de facilement : annoncer un nouveau réseau ou IP, modifier ou supprimer une route, envoyer des commandes à Exabgp( reload, show routes show neighbors et version)
\\
\\
\includegraphics[scale = 0.5]{img/erco1.JPG}
\includegraphics[scale = 0.5]{img/erco2.JPG}

\subsection{ExaBGPmon}

\includegraphics[scale = 0.40]{img/exabgpmon.png}
\includegraphics[scale = 0.40]{img/exabgpmon2.png}

\subsection{ExaBGP}
ExaBGP est un outil open source écrit en Python qui permet d'interagir avec les réseaux BGP. Le logiciel peut injecter des routes annoncés dans les réseaux. 
ExaBGP offre un API contenant plusieurs commandes afin de manipuler les routeurs BGP. On peut aller voir la liste des commandes dans l'API d'ExaBGP : 
\\
\\
\url{https://github.com/Exa-Networks/exabgp/wiki/Controlling-ExaBGP-:-interacting-from-the-API}

\subsection{Meteor.JS \cite{Meteor.JS}}
\begin{center}
\includegraphics[height=1cm]{img/Meteor-logo.png}
\end{center}

Meteor.JS est un framework open-source javascript, Node.JS qui permet l'élaboration d'une application web de type RESTful. Elle permet de développer le client et le serveur de l'application web avec le même langage.  

client javascript RESTfull, nous plus facile package, serveur cache client.


\section{Cahier des Charges}
Après avoir analysé les outils existants et aussi les besoins du client, on s'est rendu compte que le client aura besoin d'une application Web de type RESTful pour pouvoir interagir avec ExaBGP.
\newline
Par conséquent, notre application Web s'appuiera sur le framework Meteor JS. 

\subsection{Besoins fonctionnels}

\begin{itemize}
\item Ajouter/Supprimer une route (admin) : 
	\begin{itemize}
		\item L'administrateur réseau peut annoncer un réseau ou une adresse IP.
        \item L'administrateur réseau peut ainsi supprimer une adresse IP ou un réseau.
        \item La possibilité d'attribuer une communauté à un router ou bien un réseau lors de l'ajout.
        \item Les opérations de la suppression et de l'ajout seront réservées à l'administrateur de l'application.
	\end{itemize}
    
\item Vérifier le bon fonctionnement ExaBGP (admin) :
	\begin{itemize}
		\item À l'aide d'un élément dynamique, l'administrateur peut observer l'état du ExaBGP avant de lancer des commandes.
        \item Envoyer un message d'avertissement(popup)quand ExaBGP est en panne ou ne tourne pas.
	\end{itemize}
    
\item Exécuter les différentes commandes de l'API de ExaBGP
	\begin{itemize}
		\item (Utiliser la technique du Black hole selon une source IP ou bien une destination) : expliquer ça à l'aide d'un UML de Séquence.
        \item Les différentes commandes d'ExaBGP.
	\end{itemize}	

\item Relancer ExaBGP (admin)

\item Rechercher des routes selon leurs préfixes, (adresse IP, communauté, destination) :
	\begin{itemize}
		\item Une page web dans l'application dédiée à effectuer la recherche des routes selon leurs préfixes.
	\end{itemize}

\item Lister les routes :
	\begin{itemize}
		\item L'utilisateur peut voir toutes les routes qui existent dans la base de données.
        \item Le résultat sera découpé en plusieurs pages web pour facilité la lisibilité.
	\end{itemize}
\end{itemize}    

\subsection{Besoins non fonctionnels}

\begin{itemize}
\item Une base de données stockant les informations des routes : 
	\begin{itemize}
		\item Utiliser une base de données NoSQL(MongoDB)
	\end{itemize}
    
\item Certains packages de meteor.js
\item Synchronisation du serveur web avec ExaBGP
\item Sécurité, fiabilité (https...)
\item Interface différente pour l'admin et l'utilisateur anonyme
\end{itemize}
\section{Schéma de structure}

\begin{figure}[h]
\includegraphics[scale = 0.75]{img/Schema_de_structure.png}
\caption{Schéma de structure}
\end{figure}
\section{Diagramme de cas d'utilisation}
%détailler un peu
\begin{center}
\begin{figure}[h]
\includegraphics[scale=0.5]{img/diagramme_utilisation.png}
\caption{cas d'utilisation}
\end{figure}
\end{center}


\section{Diagramme de séquence}

Nous représentons le scénario de la déviation d'une route vers un trou noir en utilisant le diagramme de séquence.
\newline
L'utilisateur de l'application web dans cet exemple est l'administrateur réseaux. On suppose que l'administrateur réseaux veut dévier une route par son adresse IP source. 
\newline
Lorsque l'opération est terminé, donc l'application web confirme que c'est fait, la route est ensuite serait stockée dans la base de données afin que le client puisse aller consulter les routes qui ne sont pas accessibles.
\\
\\

\includegraphics[scale = 0.5]{img/seqDiagramme}

\section{Maquette de l'interface web}

\includegraphics[scale=0.5]{img/maquette.png}
\\
Le bouton connexion permet à l'administrateur de se connecter en entrant son login/email et son mot de passe.\\
Le bouton "API" sert à rediriger vers l'API de ExaBGP à l'url suivant:
\\\url{https://github.com/Exa-Networks/exabgp/wiki/Controlling-ExaBGP-:-interacting-from-the-API}\\
La fonctionnalité "Statut ExaBGP" sert à afficher si ExaBGP fonctionne ou non.\\
Le bouton liste de commande permet de sélectionner la commande d'ExaBGP voulue.\\
En fonction de la commande sélectionnée un formulaire spécifique est affiché. Cette partie sera affichée dans les cas où la commande a besoin de paramètre.\\
Le bouton "lancer commande" permet d’exécuter la commande sur ExaBGP ou sur la base de donnée.\\
Une zone dédiée au retour de message de la commande lancé. Le message devra être le plus détaillé possible surtout pour les messages d'erreur d'une fonction qui s'est mal exécutée.\\
Une barre de recherche qui permettra à l'utilisateur ou à l'administrateur d'effectuer une recherche par préfixe d'adresse ip.\\
Une zone d'affichage qui listera la liste des routes. Initialement cette liste affichera les plus récentes routes stocké dans la base de donnée, mais si l'utilisateur effectue une recherche, c'est cette recherche qui sera affichée dans cette zone.
\section{Diagramme de Gant}
%(2) repartir les taches
\begin{table}[h]
\begin{turn}{90}
\centering
\begin{tabular}{|p{2.5cm}|*{10}{p{1.0cm}|}}

\hline
 & semaine 5: 5 fév / 9 fév & semaine 6: 12 fév / 16 fé & semaine 7: 19 fév / 23 fév & semaine 8: 26 fév / 2 mars & semaine 9 : 27 fév / 3 mars & semaine 10: 5 mars / 9 mars & semaine 11: 12 mars / 16 mars & semaine 12: 19 mars / 23 mars & semaine 13: 26 mars / 30 mars & semaine 14: 2 avr / 6 avr \\
 \hline
Mise en place de machine virtuelle pour test & X & X & X & & & & & & &\\
\hline
connecter ExaBGP au serveur web & & & X & & & & & & &\\

\hline
état de ExaBGP & & & & X & & & & & &\\
& & & & & & & & & &\\
\hline
Connexion administrateur & & & & X & & & & & &\\
\hline
Afficher le message de retour & & & & & X & & & & &\\
\hline
annoncer une route & & & & & X & & & & &\\
\hline
supprimer une adresse IP ou réseau & & & & & X & & & & & \\
\hline
Lister les routes & & & & & X & & & & &\\
& & & & & & & & & &\\
\hline
dévier une route & & & & & X & X & & & &\\
& & & & & & & & & &\\
\hline
Recherche IP, Route par préfixe & & & & & & X & X & X & & \\
\hline
Relancer ExaBGP & & & & & & & & X & &\\
\hline
Exécuter les différentes commandes d'ExaBGP & & & & & & & & X & X & X\\
\hline
\end{tabular}
\end{turn}
\end{table}

\addcontentsline{toc}{section}{Bibliographie et Références}
\bibliography{biblio}

\end{document}